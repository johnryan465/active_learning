\documentclass[12pt, a4paper]{report}


\usepackage[T1]{fontenc}
\usepackage[latin1]{inputenc}
\usepackage[english]{babel}
\usepackage{siunitx}
\usepackage{graphicx}
\usepackage{tipa} % for the \ark{} command
\usepackage{graphics} % for pdf, bitmapped graphics files
\usepackage{times} % assumes new font selection scheme installed
\usepackage{amsmath}
\usepackage{latexsym}
\usepackage{amscd}% for commutative diagrams
\usepackage{mathrsfs} %this package is for the script font \mathscr
\usepackage{relsize}
\usepackage{delarray}
\usepackage{pstricks}
\usepackage{theorem}
\usepackage{changepage}
\usepackage{euscript}
\usepackage{textcomp}
\usepackage{esvect}
\usepackage{parskip}
\usepackage{placeins}
\usepackage{subfigure}
% \usepackage{subcaption}
\usepackage{array}
\usepackage{delarray}
\usepackage{stmaryrd}
\usepackage{fancyhdr}
\usepackage{graphpap}
\usepackage{makeidx}
\usepackage{enumerate}
\usepackage{esint}
\usepackage{datetime}
\usepackage{caption}
\usepackage{smartdiagram}
\usesmartdiagramlibrary{additions}
%Set Abstract Page
\usepackage{abstract}
\setlength{\absleftindent}{-5mm}
\setlength{\absrightindent}{-5mm}

%Colour definitions - put before TikZ
\usepackage{color}
\definecolor{igreen}{rgb}{0.0, 0.56, 0.0}
\usepackage{xcolor, colortbl}
\colorlet{gred}{-red!75!green!65!}
\colorlet{mamber}{-red!75!green!15!blue!50!}
\colorlet{grown}{-red!75!blue!20!green}
\colorlet{bled}{-red!85!blue!40!green!45!}
\colorlet{waters}{cyan!25} % Define color for the water
\colorlet{water}{cyan!25!green!20!} % Define color for the water
\definecolor{grin}{HTML}{00F9DE}
\usepackage{rotating}
\providecommand{\keywords}[1]{\textbf{\textit{Keywords---}} #1}

% For faint dotted table line
\usepackage{arydshln}
\setlength{\dashlinedash}{.4pt}
\setlength{\dashlinegap}{.8pt}

\usepackage{booktabs}
\usepackage{graphicx}
\usepackage{tikz}
\usepackage{tikz-3dplot}
\usetikzlibrary{
arrows,
arrows.meta,
automata,
backgrounds,
calc,
decorations,
decorations.pathmorphing,
decorations.pathreplacing,
decorations.fractals,
external,
fit,
matrix,
petri,
positioning,
shadows,
shapes,
shapes.multipart,
topaths,
intersections
}
\usepackage{eso-pic}
\def\ba{\begin{array}}
\def\ea{\end{array}}
\def\beann{\begin{eqnarray*}}
\def\eeann{\end{eqnarray*}}
\def\bea{\begin{eqnarray}}
\def\eea{\end{eqnarray}}
\def\bsy{\boldsymbol}
\def\gray#1{{\color{gray}#1}}

%% COUNTERS
\setcounter{MaxMatrixCols}{20}
\renewcommand{\thesection}{\arabic{section}}
\renewcommand{\thesection}{\thechapter.\number\numexpr\value{section}}
\renewcommand{\thesubsection}{\thesection.\number\numexpr\value{subsection}}
%%For changemargin
\def\quote{\list{}{\rightmargin\leftmargin}\item[]}
\let\endquote=\endlist 
\def\changemargin#1#2{\list{}{\rightmargin#2\leftmargin#1}\item[]}
\let\endchangemargin=\endlist 
\makeatletter
\newlength\qvec@height
\newlength\qvec@depth
\newlength\qvec@width
\newcommand{\qvec}[2][]{
    \settoheight{\qvec@height}{$#2$}
    \settodepth{\qvec@depth}{$#2$}
    \settowidth{\qvec@width}{$#2$}
  \def\qvec@arg{#1}
  \raisebox{.2ex}{\raisebox{\qvec@height}{\rlap{% 
    \kern.05em
    \begin{tikzpicture}[scale=1,shorten >=-3pt,shorten <=-3pt]
    \pgfsetroundcap
    \coordinate (Stx) at (.05em,0) ;
		\coordinate (Arx) at (\qvec@width-.05em,0) ;
    \draw[->](Stx) to[bend left] (Arx);
    \end{tikzpicture}
  }}}
  #2
}
\makeatother
\makeatletter
\newlength\pvec@height
\newlength\pvec@depth
\newlength\pvec@width
\newcommand{\pvec}[2][]{
    \settoheight{\pvec@height}{$#2$}
    \settodepth{\pvec@depth}{$#2$}
    \settowidth{\pvec@width}{$#2$}
  \def\pvec@arg{#1}
  \raisebox{.2ex}{\raisebox{\pvec@height}{\rlap{% 
    \kern.05em
    \begin{tikzpicture}[scale=1,shorten >=-3pt,shorten <=-3pt]
    \pgfsetroundcap
    \coordinate (Stx) at (.05em,0) ;
		\coordinate (Arx) at (\pvec@width-.05em,0) ;
    \draw[->](Stx) to[bend right] (Arx);
    \end{tikzpicture}
  }}}
  #2
}
\makeatother
\makeatletter
\newlength\vvec@height%
\newlength\vvec@depth%
\newlength\vvec@width%
\newcommand{\vvec}[2][]{%
  \ifmmode%
    \settoheight{\vvec@height}{$#2$}%
    \settodepth{\vvec@depth}{$#2$}%
    \settowidth{\vvec@width}{$#2$}%
  \else 
    \settoheight{\vvec@height}{#2}%
    \settodepth{\vvec@depth}{#2}%
    \settowidth{\vvec@width}{#2}%
  \fi%
  \def\vvec@arg{#1}%
  \def\vvec@dd{:}%
  \def\vvec@d{.}%
  \raisebox{.2ex}{\raisebox{\vvec@height}{\rlap{%
    \kern.05em%
    \begin{tikzpicture}[scale=1]
    \pgfsetroundcap
    \draw (.05em,0)--(\vvec@width-.05em,0);
    \draw (\vvec@width-.05em,0)--(\vvec@width-.15em, .075em);
    \draw (\vvec@width-.05em,0)--(\vvec@width-.15em,-.075em);
    \ifx\vvec@arg\vvec@d%
      \fill(\vvec@width*.45,.5ex) circle (.5pt);%
    \else\ifx\vvec@arg\vvec@dd%
      \fill(\vvec@width*.30,.5ex) circle (.5pt);%
      \fill(\vvec@width*.65,.5ex) circle (.5pt);%
    \fi\fi%
    \end{tikzpicture}%
  }}}%
  #2%
}
\makeatother
\def\ba{\begin{array}}
\def\ea{\end{array}}
\def\beann{\begin{eqnarray*}}
\def\eeann{\end{eqnarray*}}
\def\bea{\begin{eqnarray}}
\def\eea{\end{eqnarray}}
\def\bsy{\boldsymbol}
\def\gray#1{{\color{gray}#1}}
\usepackage{titlesec}
\usepackage{multirow}
%To reference within text
\usepackage{hyperref}
\usepackage{apacite}
\usepackage{lipsum}
\usepackage{tikz-cd}
\usepackage{float}
\usepackage{titling}
\usepackage{epigraph}
\usepackage[title, titletoc]{appendix}
\setlength\epigraphwidth{8cm}
\setlength\epigraphrule{0pt}

\titleformat{\chapter}{\normalfont\LARGE}{\thechapter\,$\vert$}{20pt}{\LARGE}{\setcounter{chapter}{0}}
\setlength{\headheight}{15pt}
\titlespacing*{\chapter}{0pt}{-70pt}{40pt} %Move chapter titles up
% Title page logos:
\makeatletter
\newcommand\BackgroundPicturea[3]{
	\setlength{\unitlength}{1pt}
	\put(0,\strip@pt\paperheight){
		\parbox[t]{\paperwidth}{
			\vspace{#2}\hspace{#3}
			\mbox{\includegraphics[scale=0.5]{#1}}
}}}
\newcommand\BackgroundPictureb[3]{
	\setlength{\unitlength}{1pt}
	\put(0,\strip@pt\paperheight){
		\parbox[t]{\paperwidth}{
			\vspace{#2}\hspace{#3}
			\mbox{\includegraphics[scale=0.3]{#1}}
}}}
\makeatother
% For my abbreviations
\newcommand{\abbrlabel}[1]{\makebox[3cm][l]{\textbf{#1}\ \dotfill}}
\newenvironment{abbreviations}{\begin{list}{}{\renewcommand{\makelabel}{\abbrlabel}}}{\end{list}}
% Line Spacing
\usepackage{setspace}
\setstretch{1.5}
%Set of command is for the changemargin environment
\def\quote{\list{}{\rightmargin\leftmargin}\item[]}
\let\endquote=\endlist 
\def\changemargin#1#2{\list{}{\rightmargin#2\leftmargin#1}\item[]}
\let\endchangemargin=\endlist
%Replace Contents to Table of Contents	
\addto\captionsenglish{
	\renewcommand{\contentsname}%
	{Table of Contents}
	\setcounter{tocdepth}{3}% Include \subsubsection in ToC
	\setcounter{secnumdepth}{3}% Number \subsubsection in ToC
	}
\renewcommand{\listfigurename}{List of Figures}
\renewcommand{\listtablename}{List of Tables}

\theoremstyle{definition}
\newtheorem{definition}{Definition}[section]

\hypersetup{pdftitle = Project Report, pdfauthor = {First Last}, pdfstartview=FitH, pdfkeywords = essay, pdfpagemode=FullScreen, colorlinks, anchorcolor = red, citecolor = blue, urlcolor=blue, filecolor=green, linkcolor=red, plainpages=false}
%%%%%%%%%%%%%%%%%%%%%%%%%%%%%%%%%%%%%%%%%%%%%%%%%%%%%%%%%%%%%%%%%%%%%%%
\pagestyle{fancy}
\rhead{Christ Church}
\chead{}
\lhead{University of Oxford}
\lfoot{\date{}}
\cfoot{}
\rfoot{\thepage}
% Top and Bottom Line Rules
\renewcommand{\headrulewidth}{0.4pt} %0.4pt
\renewcommand{\footrulewidth}{0.4pt}
\fancyheadoffset{9pt}
\fancyfootoffset{9pt}
% Line spacing
\renewcommand{\baselinestretch}{1.5} %1.5
%%%%%%%%%%%%%%%%%%%%%%%%%%%%%%%%%%%%%%%%%%%%%%%%%%%%%%%%%%%%%%%%%%%%%%%
\date{}

\title{Active Learning}
\author{\\ \Large{John Ryan}
\\ Christ Church
\\
\\
\\
\\ University of Oxford
\\
\\ \\
Trinity 2021
}
%%%%%%%%%%%%%%%%%%%%%%%%%%%%%%%%%%%%%%%%%%%%%%%%%%%%%%%%%%%%%%%%%%%%%%%
\begin{document}
% Adjust logo positions here
\AddToShipoutPicture*{\BackgroundPicturea{Logos/logo2.png}{0.7in}{5.8in}}

\thispagestyle{headings}
	\maketitle
\FloatBarrier
\pagenumbering{roman}


\thispagestyle{empty}
\begin{abstract}

Active Learning is an use

\keywords{Keyword1 - Keyword2 - Keyword3}
% \vspace{-10mm} %To remove added white space after
\end{abstract}
\tableofcontents
\thispagestyle{plain}
\listoffigures
\listoftables

\chapter*{List of Abbreviations}
\begin{abbreviations}
    \item[GP] Gaussian Process
    \item[GPC] Gaussian Process Classification
    \item[BALD] Bayesian Active Learning by Disagreement
    \item[ELBO] Evidence Lower Bound
    \item[ML] Machine Learning
\end{abbreviations}

\chapter{Introduction}
\pagenumbering{arabic}

Active learning is an area of machine learning where we are attempting to select which unlabelled datapoints we believe would be best to learn the label of. It is important because in many important use cases (for example, medical data), it is often either quite costly or just simply impossible to collect an incredibly large dataset. However current approaches to do it suffer from issues related to quality of aquired datapoints, model requirements and computational issues.

Bayesian ML models are the standard ways of building machine learning models which can calculate uncertainty well, however many of the standard approaches to doing so (MC Dropout) can give suboptimal performance, for example requiring a substantial amount of sampling. Gaussian Processes are a ML approach which give very good uncertainty performance, and new results in the field [DUE] have shown successful training of Variational Gaussian Processes with feature extractors and demonstrate that they maintain good uncertainty performance.

\section{Motivation}

Active Learning is an active area of Machine Learning where we are trying to select (unlabelled) data points in an attempt to maximise some objective.
\section{Aim and Objectives}

The aims of this report is to

\begin{itemize}
    \item Give an outline of the various methods for Active Learning which have been published
    \item Compare these methods in a reproducible and standardised fashion
    \item Extend and combine some of these approaches with other ML research to improve Active Learning performance in certain situations.
\end{itemize}
\section{Project Report Outline}
The remainder of this report is organised as follows:
\begin{itemize}
    \item[] \textbf{Chapter} \hyperref[Chap2]{\textbf{2}} --- Defines active learning, and introduces the different active learning methods from the literature.
    \item[] \textbf{Chapter} \hyperref[Chap3]{\textbf{3}} --- contains the performance of these methods compared
    \item[] \textbf{Chapter} \hyperref[Chap4]{\textbf{4}} --- introduces Gaussian Process based models
    \item[] \textbf{Chapter} \hyperref[Chap5]{\textbf{5}} --- Combines BatchBALD and GPC
\end{itemize}
>

\chapter{Active Learning Methods}
\label{Chap2}

In a common active learning formulation we have the following setup.

\begin{itemize}
    \item $X_{pool}$ this is the distribution of the dataset which we have to work with in our data pool.
    \item $X_{true}$ this is the true real world distribution.
\end{itemize}

A common (and sensible) assumption is to assume that these distributions are the same.

Our machine learning models which we use to model a problem constrain the set of possible functions which we can represent and learn, this is how we imbue the problem with our prior beliefs about the nature of the problem.

- There are hard constraints (eg clipping the output of the model), which even given an unlimited amount of data our model can not possibly "learn around"
- There are softer constraints (priors over weights in a layer), which our model should be able to learn correctly given sufficient new data even if our prior is poor*.

This paramaterisaton of our models is very important, this is an assumption we are making about our problem.

An important observation is that the dataset we are using for training is (unless performing very simple active learning approaches eg random acquisition) not going to be the same as the true dataset which we are working with. 

This violates a very common assumption that we assume for convergence in many ML methods, and is something worth keeping in mind. This statistical bias is looked at in the following paper \cite{farquhar2021statistical}.

When working with Active Learning we are also normally working with much smaller datasets than is standard in ML, our models which are normally over paramatised to begin with become even more so, this can lead to training issues.

\section{Random}
\section{Uncertainty}
\section{Entropy}
\section{BALD}
Bayesian Active Learning by Disagreement (BALD) \cite{houlsby2011bayesian} is an active learning method which is based on selecting datapoints which maximise the mutual information between the datapoints class output and the models parameters. The paper details how we can instead of computing this in the paramater space which would be computationally prohibative in many cases, we can compute the same objective in the output space.

$${\arg\,\max}_x H \left[ \theta | D \right] - \mathbf{E}_{y \sim p(y | x, D)} \left[ H\left[ \theta | y, x, D\right]\right]$$

$${\arg\,\max}_x H \left[ y | x, D \right] - \mathbf{E}_{\theta \sim p(\theta | D)} \left[ H\left[ y | x, \theta \right]\right]$$


An issue which we run into when using BALD in the case where we want to select multiple datapoints to aquire at the same time is the fact that selecting the $n$ points with the greatest individual information content will not necessarily give you the $n$ points which jointly have the greatest information content. This can be made very obvious by considering the case when there is duplicates in the dataset. If the duplicated point is in the top n points as decieded by the BALD score, having both of the points in your aquired batch will give you no more information than just having one of them, you would be better served by selecting any other point instead of these points.

\section{BatchBALD}
BatchBALD \cite{kirsch2019batchbald}, is an extension of BALD designed especially to deal with the issues which we run into when using BALD to aquire batches of points.

To do this the objective function is modified, the original BALD objective is retained by setting $n = 1$ .

The authors of the paper prove that the objective function that they specify is \textit{submodular}, which enables a greedy $1 - \epsilon$ approximation algorithm for selecting the optimium batch.

The BatchBALD score function is as follows.

$$s_{BatchBald} (x_1, \ldots, x_n) = H(y_1, \ldots, y_n) - E_{p(f)}\left[H(y_1, \ldots, y_n | f)\right]$$



As the variables are independent conditioned on the value of the function, the right hand side of the equation factors into a sum over conditional expectations for each individual datapoint. The computationally difficult part of this objective is the computation of the joint entropy as this does not factor similarly. In fact, the summation required to compute this value is exponential in the number of datapoints we are currently considering. The authors of the paper address this issue by using sampling to estimate the joint entropy for larger batch sizes. The BatchBALD paper uses Bayesian Neural networks (BNNs) as their model. They are then able to sample from the weight distribution once to obtain a function draw for all possible inputs. This is not an ability we have when using GPCs, which we will come back to later.



\chapter{Gaussian Process Based Models}
\label{Chap4}


\section{Gaussian Processes}


\begin{definition}[Gaussian Process]
    A Gaussian processes is a collection of random variables, any finite number of which have a joint Gaussian distribution. \cite[]{rasmussen2003gaussian}
\end{definition}


We can completely specify a Gaussian Process over a space $\chi$ by giving a mean function $m(\cdot), m : \chi \rightarrow R$, and covariance (kernel) function $k(\cdot, \cdot), k: \chi \times \chi \rightarrow R$.


For $k$ to be a valid covariance funciton we need for any evaluation of the covariance matrix to be a symetric positive definite matrix (as all covariance matrices of random variables must be S.P.D).

\begin{definition}[Positive Definte Matrix]
    A positive definite matrix is a matrix which has positive eiginvalues.
\end{definition}

\subsection{As distributions over functions}

We can view a Gaussian Processes as a distribution over functions. The space of functions which we can represent by a particular Gaussian Process is paramaterised by the kernel function which we are using.

We can understand this by investigating the interpretation of kernels as Repreducing Kernel Hilbert Spaces.





Gaussian Processes have a number of useful properties such as being closed under conditioning and marginalisation.

For the covariance function to define a valid MVN for all inputs the covariance function must be positive definite.

A positive definite kernel function can be represented by a feature space where the value of the kernel between 2 values can be computed by taking the inner product of the feature space of the 2 values.

Certain kernel functions such as the radial basis kernel, is represented by an infinite dimensional feature space. This infinite dimensional feature space enables it to model arbitarily complex functions. Polynomial or linear kernels do not enable this level of expressibility.

We can use Bayes rule to update the covariance and mean function from our prior, to the postior given noisy of exact data. The complexity of doing this (given arbitary kernels) is bounded by matrix inversion and  multiplication of a matrix of size of the number of datapoints in which we are using. The complexity of which is sub cubic, however in general the matrix multiplication algorithms we use practically are of the cubic variety).

This can be done as follows: 

(TODO) put update equations.

\subsection{Scaling to large data}

Unlike other methods we use often in machine learning, such as neural networks which have linear complexity in the size of the dataset we have superlinear complexity.

\subsubsection{Inducing Points}

In inducing point models, we instead of using all of the datapoints in our model. We instead use a smaller number of datapoints as an approximation to the true model. Various approaches exist for selecting the points we keep in our approximation.

One of the most sucessful approaches is to treat these inducing points as model parameters which we attempt to learn by minimising some objective.

Minimising the evidence lower bound (ELBO) is a common and principled object for this task.

Minimising this objective minimises the KL Divergence between the true output distribution and the approximate distribution we are learning.

\subsubsection{Random Fixed Features}

\section{Gaussian Process Classification}


To use Gaussian Processes for classification we have several issues to overcome.

The output of a Gaussian Process is a MVN, this is a continous distribution over the real space, however in the classification setting we wish to have a categorical output. Which means we need to have a distribution over the the simplex.

We can in theory use any link function to convert a sample from our MVN in $R^n$ to a sample over the simplex, different link functions have different properties computationally and statistically.

\subsection{Link Functions}

The link function between the latent function (our GP) and the output of our classifier is again another parameter of the model.

\subsubsection{Logit}

The logit is the standard link function for the vast majority of GPC use cases. The use of the logit implicitly makes our GPC model obey the independence of irrelevant alternatives axiom from decision theory. 


$$p\left( y = k | f\right) = \frac{e^{f^T w_k}}{\sum_{i=1}^{K} e^{f^T w_i}}$$
\subsubsection{Probit}



\section{vDUQ / DUE}

vDUQ / DUE \cite{vanamersfoort2020uncertainty} is a recent model which deals with some of the issues with using deep kernel learning with GPCs. DUE uses an inducing point GP to enable it to scale to large amounts of data. To bound the lipschitz constant of the model, DUE uses spectral normalisation in the feature extractor of the model.
To enforce sensitivity to chages in the input, 
\subsection{Spectral Normalisation}

\chapter{Combination}
\label{Chap5}

\section{Investigation of Exact Computation}

If we use Probit as our link function the joint entropy can be computed exactly (unlike in the general case), however the complexity of this exact computation is very high.

To perform exact inference of the probit of a Gaussian Classifier we must perform integration of a Gaussian.

We wish to find the probability that a certain element of a MVN is the largest element. This corresponds to integrating the Gaussian over the subset of the space where this coordinate is the largest.

$$P(X_C > X_1 \land \ldots \land X_C > X_n) = \int_{x_c \in \{ -\infty, \infty \} } \ldots \int_{x_n \in \{ -\infty , x_c \}} p(x) dx$$

This can be interpreted as integrating on one side of several hyperplanes.

These planes are of the form $x_c = x_i$

With the linear transformation of $T = I - e_c 1^T$ we get to transform this integral.

$Z = TX$.

$Z \sim N(T \mu, T \Sigma T^T)$

We drop the original index (as we are marginalising over it) and it is degenerate. We could alternatively make T a projection into a c-1 dimensional subspace.

If we are doing this over multiple datapoints, we can transform each of the subsets of variables via the method above.

After performing this transformation we have a standard orthant integral which we can compute. Algorithms exist for computing this exactly in $O(p^2 2^p)$ using recursive integration and subspace projection. \cite{orthant}


However the number of dimensions which we are performing this integration over is (number of variables) * (number of categories - 1). With the exponential complexity this an infeasible method of computation.


\section{Sampling}

Given the difficultly of performing exact computation of the quantities required for the BatchBALD objective, the general approach is to use methods which rely on sampling.

In comparision with the approach performed in the BatchBALD paper where sampling is done over the weights of the Bayesian Neural Network, when we have a Gaussian Process as our output. It is not as straightforward to take a sample function from our postior. If we are to take a sample function output over a set out points, we would need to compute the joint distribution of these points and then take a sample from this distribution and pass this through our link function.
The complexity of performing this is cubic in the size of the pool, this complexity is unacceptably high as we are often working with quite large pool sizes.

This means that we can not use the same computational tricks used previously to speed up these calculations, however we can take advantage of other properties of Gaussian Processes to make some improvements in this aspect.


\subsection{}


\chapter{Methodology}
\label{Chap6}

All of the experiments referencedx here can easily be replicated using the codebase for this project which is avaiable here.

The code uses the frameworks \cite[Pytorch]{NEURIPS2019_9015} and \cite[GPytorch]{gardner2018gpytorch}.



\section{Replication}

To begin with replication of some of the experiments in the BatchBALD to valididate the experimental framework and to have points of comparision to work with.




\section{DUE}

\subsection{Model Outline}


\subsubsection{Training issues}

When training DUE/vDUQ based models on extremely small datasets, an issue which begins to become apparent is the issue of inducing point initialisation. As feature extractor is randomly initialised, the initial inducing points we get obtain by using standard initial inducing point initialisation procuedures can lead to poor training dynamics.





\subsection{Sampling Functions}

As mentioned earlier in the paper sampling a function over the entire pool is computationally prohibative, however we can take a small enough subset of the pool over which we can calculate the joint distribution and take sample function draws.
We can then use these as a numerical test to valididate that the more computationally efficent methods we develop are giving correct results.


\chapter{Experiments}

\section{Replication}

\subsection{Random}

\subsection{BALD}

\subsection{BatchBALD}

\subsection{Predictive Entropy}

\section{DUE}

To attempt to keep the experiments as comparible as possible, the feature extractor architecture used for DUE for this problem was the same architecture as the BNN with minimal modifications to satify the constraints of DUE. This primarily consisted of the additon of residual connections to the models, and the removal of the dropout layers.

While DUE has no problems in learning MNIST with more expressive models, eg more traditional ResNets. When using these the performance on very small datasets was reduced significantl.

\subsection{Training on small datasets}


\subsection{BALD}

\subsection{BatchBALD}

\subsection{Predictive Entropy}






\renewcommand{\bibname}{Bibliography}
\bibliographystyle{apacite}
\bibliography{Bibliography.bib}

\begin{appendices}
\chapter{Appendix Example}
\end{appendices}

\end{document}